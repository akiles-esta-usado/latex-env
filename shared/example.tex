\documentclass[12pt]{article}
\usepackage{amsmath} % for the equation* environment

\begin{document}

\section*{Notes for My Paper}

Don't forget to include examples of topicalization.
They look like this:

\subsection*{How to handle topicalization}

I'll just assume a tree structure


\subsection*{Mood}

Mood changes when there is a topic, as well as when
there is WH-movement.  \emph{Irrealis} is the mood when
there is a non-subject topic or WH-phrase in Comp.
\emph{Realis} is the mood when there is a subject topic
or WH-phrase.

The well known Pythagorean theorem \(x^2 + y^2 = z^2\) was
proved to be invalid for other exponents.
Meaning the next equation has no integer solutions:

\[ x^n + y^n = z^n \]

\section{Another example}

\noindent Standard \LaTeX{} practice is to write inline math by enclosing it between \verb|\(...\)|:

\begin{quote}
    In physics, the mass-energy equivalence is stated
    by the equation \(E=mc^2\), discovered in 1905 by Albert Einstein.
\end{quote}

\begin{quote}
    In physics, the mass-energy equivalence is stated
    % by the equation $E=mc^2$, discovered in 1905 by Albert Einstein.
    by the equation \(E=mc^2\), discovered in 1905 by Albert Einstein.
\end{quote}

\noindent Or, you can use \verb|\begin{math}...\end{math}|:

\begin{quote}
    In physics, the mass-energy equivalence is stated
    by the equation \begin{math}E=mc^2\end{math}, discovered in 1905 by Albert Einstein.
\end{quote}


\section{Another Example 2}

The mass-energy equivalence is described by the famous equation

\[E=mc^2\]

discovered in 1905 by Albert Einstein.
In natural units (\(c\) = 1), the formula expresses the identity

\begin{equation}
    E=m
\end{equation}


\section*{Example with amsmath environment}


This is a simple math expression \(\sqrt{x^2+1}\) inside text.
And this is also the same:
\begin{math}
    \sqrt{x^2+1}
\end{math}
but by using another command.

This is a simple math expression without numbering
\[\sqrt{x^2+1}\]
separated from text.

This is also the same:
\begin{displaymath}
    \sqrt{x^2+1}
\end{displaymath}

\ldots and this:
\begin{equation*}
    \sqrt{x^2+1}
\end{equation*}

\end{document}